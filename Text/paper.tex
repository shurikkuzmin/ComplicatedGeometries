\documentclass{article}
\title{Conditions for successful wetting during collision between emulsion droplets and curved substrate}
\begin{document}
\maketitle
\section{conclusions}
The work presented investigates the interaction between emulsion droplets and curved substrates to understand the critical factors controlling the efficiency and the kinetics of the wetting process during collision. We use 2D Lattice Boltzmann methods to simulate spherical emulsion droplets colliding on a curved "cylindrical" substrate. The properties of the substrate are set by controlling its curvature and the three phase contact angle with the emulsion droplet and the surrounding fluid. The emulsion droplet is characterized by its size, viscosity and interfacial tension with the surrounding fluid. As expected we observe that the curvature of the droplet or substrate and the kinetics of the collision process does not effect the equilibrium three phase contact angle after wetting. However a detailed analysis of the collision process reveals that droplets deformation at wetting critically depends on the collision speed and relative dimension of droplet and substrate. Large droplets approaching the substrate very slowly deform very mildly before wetting while during fast collision a higher deformation is achieved before wetting occurs. For small droplets the same trend is found but overall smaller deformation are observed. The results can be rationalized if we consider a hydrodynamic disjoining pressure that retard contact between droplets and substrate. Based on this finding we can map the optimal conditions to achieve successful wetting, deposition and retention of an emulsion droplet on a substrate. This results can be used to guide the design of emulsions used as actives delivery vehicles. Emulsion size and interfacial properties can be tuned to optimize deposition efficiency on the relevant substrates based on the hydrodynamic parameters of the collision process. Such information are of important practical utility in farmaceutical and cosmetic application in which emulsion droplets are often used to deposit active ingredient on tissues or fibers.  
\end{document}
