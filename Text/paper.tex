\documentclass{article}
\usepackage{amsmath,amssymb}
\usepackage{bm}
\usepackage[numbers,sort&compress]{natbib}
\usepackage{color}
\usepackage{graphicx}

\newcommand{\Ca}{\mathrm{Ca}}
\newcommand{\beq}{\begin{equation}}
\newcommand{\feq}{\end{equation}}
\newcommand{\beqal}{\begin{equation}\begin{aligned}}
\newcommand{\feqal}{\end{aligned}\end{equation}}

\title{Conditions for successful wetting during collision between emulsion droplets and curved substrate}
\begin{document}
\maketitle


\section{Binary-liquid lattice Boltzmann model}
{\color{red} This text was taken from the old text and needs to be modified}

The lattice Boltzmann equation (LBE) operates on a rectangular grid representing the
physical domain. It utilizes
probability distribution functions (also known as particle populations)
containing information about
macroscopic variables, such as fluid density and momentum. LBE consists of
two parts: a local collision step, and a propagation step which transports
information from one node to another along some 
directions specified by the discrete velocity set.
The LBE is typically implemented as follows:
\begin{equation}
\label{standard:implementation}
\begin{aligned}
&f_i^{*}(\bm{x},t)=\omega f_i^{eq}(\bm{x},t)-(1-\omega) f_i(\bm{x},t) +
F_i,&&\text{ collision step}\\
&f_i(\bm{x}+\bm{c_i},t+1)=f_i^{*}(\bm{x},t),&&\text{ propagation step}, 
\end{aligned}
\end{equation}
where $f_i$ is the probability distribution function in the direction $\bm{c_i}$,
 $f_i^{eq}$ is the equilibrium probability distribution function, $\omega$ is the
relaxation parameter, and $F_i$ is the external force population. The force population
represents an external physical force and is implemented in the current work using the scheme
outlined in \citet{guo}.

The binary fluid LB model is
based on a free-energy functional \cite{swift,landau}, and operates with two
sets of populations: one to track the pressure and the velocity fields, and another to represent the
phase field $\phi$ indicating the gas or liquid.

The model we use is a two-dimensional nine-velocity (D2Q9) model,
with equilibrium populations \cite{pooley-contact}:
\begin{equation}
\label{set:equilibrium:binary}
\begin{aligned}
&f_i^{eq}&&=w_i 
\biggl(3
p_0 - k \phi \Delta \phi
+\rho\frac{u_{\alpha}c_{i\alpha}}{c_s^2}+\rho \frac{Q_{i\alpha\beta}u_{\alpha } u_ {
\beta}}{2 c_s^4}\biggr)\\
&&&+k\bigl(w_i^{xx} (\partial_x \phi)^2+w_i^{yy} (\partial_y \phi)^2 +w_i^{xy} \partial_x
\phi \partial_y \phi \bigr), 1\leq i \leq 8\\
&f_0^{eq}&&=\rho-\sum_{i\neq0}{f_i^{eq}}\\
&g_i^{eq}&&=w_i\left(\Gamma \mu + \phi\frac{ c_{i\alpha} u_{i\alpha}}{c_s^2}+\phi
\frac{Q_{i\alpha\beta}u_{\alpha}u_{\beta}}{2 c_s^4}\right), 1\leq i \leq 8 \\
&g_0^{eq}&&=\phi-\sum_{i\neq0}{g_i^{eq}}\quad,
\end{aligned}
\end{equation}
where $\Gamma$ is the mobility parameter; the chemical potential
$\mu=-A\phi+A\phi^3-k\Delta\phi$; $k$ is the parameter related to the surface
tension; $A$ is the parameter of the free-energy model. The bulk pressure
is expressed as $p_0=c_s^2 \rho +A (-0.5 \phi^2+0.75 \phi^4)$ with
the sound speed $c_s^2=1/3$. 
Parameters specific to the D2Q9 grid are the weights
$w_i=\left\{\frac{4}{9},\frac{1}{9},\frac{1}{9},\frac{1}{9},\frac{1}{9},
\frac{1}{36},\frac{1}{36},\frac{1}{36},\frac{1}{36}\right\}$, and the tensor
$Q_{i\alpha\beta}=c_{i\alpha} c_{i\beta} - c_s^2 \delta_{\alpha\beta}$.  
Other weights 
%related to the
%inclusion of the surface tension coefficient into the equations 
are as follows:
$w^{xx}_{1-2}=w^{yy}_{3-4}=1/3$, $w^{xx}_{3-4}=w^{yy}_{1-2}=-1/6$,
$w^{xx}_{5-8}=w^{yy}_{5-8}=-1/24$, $w^{xy}_{1-4}=0$, $w^{xy}_{5-6}=1/4$ and
$w^{xy}_{7-8}=-1/4$. The set of equations (\ref{set:equilibrium:binary}) restores the
macroscopic
fluid equations as:
\begin{equation}
\begin{aligned}
&\partial_t \rho+ \partial_{\alpha} \rho u_{\alpha}=0\\
&\rho\left(\partial_t+u_{\beta}\partial_{\beta}\right) u_{\alpha}= F_{\alpha}
-\partial_{\beta}P_{\alpha \beta} +
\nu\partial_{\beta}\left(\partial_{\alpha}u_{\beta}+\partial_{\beta} u_{\alpha}\right)\\
&\partial_t \phi + \partial_{\alpha} \phi u_{\alpha}=M \partial^2_{\beta\beta} \mu,
\end{aligned}
\label{binary:fluid:system}
\end{equation}
where $\nu=c_s^2 (\tau-1/2)$ is the viscosity,
$M=\Gamma(\tau_{\phi}-1/2)$ is the mobility parameter, and $\tau=\frac{1}{\omega}$ and $\tau_{\phi}$
are the relaxation parameters of density and phase fields, 
$P_{\alpha\beta}=\Bigl(p_0-k\phi \Delta \phi -\frac{k}{2}|\nabla \phi|^2\Bigr)\delta_{\alpha\beta}
+ k \partial_{\alpha} \phi \partial_{\beta} \phi$  \cite{pooley-contact}.The interface tension value
in the framework of the binary liquid model is $\gamma=\sqrt{\frac{8 k
A}{9}}$. The inclusion of the interface tension in the momentum flux tensor is done through the
coefficients $k$, $A$ and weights $w_i^{\alpha\beta}$.

Note that the first equation of system \ref{standard:implementation} simulates the continuity and
the Navier-Stokes equations, i.e. the first two equations in (\ref{binary:fluid:system}). The second
equation
of system \ref{standard:implementation} simulates the phase governing equation, i.e. the third
equation in
(\ref{binary:fluid:system}). The system (\ref{binary:fluid:system}) allows the separation of the
liquid
phase with $\phi=1$ and a so-called gas phase with $\phi=-1$. The
relaxation time is taken as linearly dependent on the relaxation
times $\tau_{\mathrm{gas}}$ and $\tau_{\mathrm{liq}}$:
$\tau=\tau_{\mathrm{gas}}+\frac{\phi+1}{2}(\tau_{\mathrm{liq}}-\tau_{\mathrm{gas}})$. This allows
to change viscosity from the gas viscosity
$\nu_{\mathrm{gas}}=\frac{1}{3}\Bigl(\tau_{\mathrm{gas}}-\frac{1}{2}\Bigr)$ to the liquid viscosity
$\nu_{\mathrm{liq}}=\frac{1}{3}\Bigl(\tau_{\mathrm{liq}}-\frac{1}{2}\Bigr)$ while phase changes
accordingly.

While the lattice Boltzmann system has parameters such as the surface tension, the gas and liquid
viscosities, etc., those parameters are not the representative and
proportional quantities of the parameters in a physical system. The parameters of the lattice
Boltzmann are connected with the physical parameters only through the non-dimensional
numbers governing the physics of the problem. In our case, these numbers are the capillary number $\Ca$ and the
viscosity ratio $\frac{\mu_{\mathrm{liq}}}{\mu_{\mathrm{gas}}}$, which are
obtained from the physical world and then matched through the lattice Boltzmann quantities.
 The set of
the fluid and phase equations \eqref{binary:fluid:system} is valid in the lattice Boltzmann space
and in the physical domain. Therefore, one can substitute any quantity, i.e.
$U_{\mathrm{bubble}}$, in the physical units or in the lattice Boltzmann units as soon as the
capillary number is the same in both worlds.
\section{Boundary conditions}
In binary-liquid model It is known that the equilibrium contact angle $\theta_{w}$ of a droplet located at a flat surface depends on the phase gradient through the complicated relation \cite{briant-contact-line}:
\beqal
&k \partial_{\perp} \phi = - h
&\sqrt{\frac{2}{k A}}h= 2 \mathrm{sign}\biggl(\frac{\pi}{2}-\theta_w\biggr) \biggl[\cos\Bigl(\frac{\alpha}{3}\Bigr)\Bigl(1-\cos\Bigl(\frac{\alpha}{3}\Bigr)\Bigr)\biggr]\\
&\cos(\alpha)=\sin^2(\theta_w),
\feqal
where $\theta_w$ is the contact angle, $\partial_{\perp}\phi$ is the imposed phase gradient at the wall. From this equation one can find the contact angle as the function of the phase gradient at the wall, $\theta_w=\theta_w(\partial_{\perp}\phi)$.

According to works \cite{manukyan-curved,carrol-curved} the contact angle at the curved substrate should have the same value as in the flat case. This case is validation case for the lattice Boltzmann simulations. The challenge is to impose certain phase gradient at the curved substrate. The lattice Boltzmann system operates on the rectangular grid and one needs to come up with simple and elegant numerical stencil to impose the phase gradient in the direction of the normal to curved substrate.
{\color{red} Put here explanation and reference to japanese work - plus validation case with all graphs.}
We present new boundary conditions to fix the angle at the surface.

\section{Conclusions}
The work presented investigates the interaction between emulsion droplets and curved substrates to understand the critical factors controlling the efficiency and the kinetics of the wetting process during collision. We use 2D Lattice Boltzmann methods to simulate spherical emulsion droplets colliding on a curved "cylindrical" substrate. The properties of the substrate are set by controlling its curvature and the three phase contact angle with the emulsion droplet and the surrounding fluid. The emulsion droplet is characterized by its size, viscosity and interfacial tension with the surrounding fluid. As expected we observe that the curvature of the droplet or substrate and the kinetics of the collision process does not effect the equilibrium three phase contact angle after wetting. However a detailed analysis of the collision process reveals that droplets deformation at wetting critically depends on the collision speed and relative dimension of droplet and substrate. Large droplets approaching the substrate very slowly deform very mildly before wetting while during fast collision a higher deformation is achieved before wetting occurs. For small droplets the same trend is found but overall smaller deformation are observed. The results can be rationalized if we consider a hydrodynamic disjoining pressure that retard contact between droplets and substrate. Based on this finding we can map the optimal conditions to achieve successful wetting, deposition and retention of an emulsion droplet on a substrate. This results can be used to guide the design of emulsions used as actives delivery vehicles. Emulsion size and interfacial properties can be tuned to optimize deposition efficiency on the relevant substrates based on the hydrodynamic parameters of the collision process. Such information are of important practical utility in pharmaceutical and cosmetic application in which emulsion droplets are often used to deposit active ingredient on tissues or fibers.  
\bibliographystyle{unsrt}
\bibliography{paper}
\end{document}
